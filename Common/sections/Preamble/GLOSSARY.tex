

\makeglossaries


% Define acronyms
\newacronym{EMR}{EMR}{Electronic Medical Record}
\newacronym{SE}{SE}{Spherical Equiivalent}
\newacronym{oco}{OCO}{Ophthalmic Clinical Officer}
\newacronym{who}{WHO}{World Health Organisation}
\newacronym{pehp}{PEHP}{Primary Eye Health Providers}
\newacronym{iapb}{IAPB}{International Agency for the Prevention of Blindness}
\newacronym{aapos}{AAPOS}{American Association for Pediatric Ophthalmology and Strabismus}
\newacronym{mnrh}{MNRH}{Mulago National Referral Hospital}
\newacronym{opd}{OPD}{Out Patient Department}
\newacronym{referror}{RE}{Refractive Error}
\newacronym{ure}{URE}{Uncorrected Refractive Error}
\newacronym{ubos}{UBOS}{Uganda Bureau of Statistics}
\newacronym{moh}{MOH}{Ministry Of Health}
\newacronym{ngo}{NGO}{Non Goverment Organisation}
\newacronym{irb}{IRB}{Institutional Reveiw Board}
\newacronym{pec}{PEC}{Primary Eye Care}



% Define the terms
\newglossaryentry{accommodation}{
    name={Accommodation},
    description={The eye's ability to automatically change focus from seeing at one distance to seeing at another 
    }
}

\newglossaryentry{albinism}{
    name={Albinism},
    description={A genetically determined heterogeneous group of disorders of melanin synthesis in which either eye alone (ocular albinism) may be affected}
}

\newglossaryentry{amblyopia}{
    name={Amblyopia},
    description={A unilateral or bilateral decrease of visual acuity caused by form deprivation and/or abnormal binocular interaction for which no organic causes can be detected by the physical examination of the eye}
}


\newglossaryentry{bcva}{
    name={Best corrected visual acuity},
    description={The best vision you can achieve with correction (such as glasses), as measured on the standard visual acuity chart}
}
\newglossaryentry{Cataract}{
    name={Cataract},  
    description = {is the loss of transparency of the crystalline lens or its capsule where there is light scattering reduction in transparency in the lens due to disorganization of the lens fiber, or disorganization of cytoplasm within the fiber, causing scattering (Brown, 2001)}
}
\newglossaryentry{Diopter}{
    name={Diopter},
    description={is a unit by which the strength of lenses is measured.}}
    
\newglossaryentry{Emmetropia}{
    name={Emmetropia},
    description={A condition whereby with accommodation relaxed, parallel rays of light coming from infinity falls on the retina}
    }
    
\newglossaryentry{LogMAR E chart}{
    name={LogMAR E chart}, 
    description= {is an acuity chart that expresses visual acuity in terms of the logarithm of the angular limb width (in minutes of arc) of the smallest letters recognized at six meters.}
    }

\newglossaryentry{snellen chart}{
    name={Snellen chart}, 
    description= {is a tool used to measure visual acuity by presenting a series of black capitalized letters on a whiteboard in rows, with each row containing letters of decreasing size \parencite{fajrin2020electronic}}
    }
    
\newglossaryentry{Ophthalmoscope}{
    name={Ophthalmoscope}, 
    description = { is an instrument which allows for the visual examination of the external and internal structures of the eye}
}
\newglossaryentry{optometrist}{
    name={Optometrist},
    description ={
    is a health care professional who typically provide comprehensive primary eye care \parencite{masnick2004optometrist}
    }
} 
\newglossaryentry{objective refracion}{
    name={Objective Refracion},
    description = {
    refers to the measurement of a patient's refractive error using instruments or techniques that do not require their subjective feedback \parencite{chauhan2022commentary}
    }
}
\newglossaryentry{Retinoscopy}{
    name={Retinoscopy}, 
    description ={
    The process of using a retinoscope for the objective determination of the patient’s refractive status
    }
    }
\newglossaryentry{autorefractor}{
    name={Autorefractor}, 
    description ={
    is a device used in ophthalmology to objectively determine refractive errors in the eye without requiring active feedback from the patient or a highly trained practitioner.Autorefractor (FR-8900) is model used in all clinics.\parencite{sumual2023evaluasi}
    }
    }
\newglossaryentry{Retina}{
    name={Retina},
    description = {
    The sensory membrane that lines the inside of the eye; it is composed of several layers and functions as the immediate instrument of vision by receiving images formed by the lens and converting them into signals which reach the brain by way of the optic nerve
    }
    }
\newglossaryentry{Visual acuity}{
    name={Visual acuity},
    description ={
    is a fundamental measure of the eye's ability to discern fine details and is crucial in assessing visual function \parencite{snow2022visual}. It is typically evaluated using standardized charts like the Snellen chart, which tests the smallest identifiable letters at a specific distance to determine the eye's resolution power
    }
    }
\newglossaryentry{Prevalence}{
    name={Prevalence},
    description = {
    is defined as the number of cases of a disease that exist in a defined population at a specified point in time
    }
    }
\newglossaryentry{Pinhole disc}{
    name={Pinhole disc},
    description ={
    is an opaque disc with a central circular aperture of about 1mm in diameter 
    }
    }
\newglossaryentry{se}{
    name={\gls{SE}},
    description = {
    is the spherical lens which places any astigmatic eye in a condition of meridional balance \parencite{pascal1954meaning}.
    }
    } 

\newglossaryentry{ophthalmogist}{
    name={Ophthalmologist},
    description = {
        are medical doctors that further specialize in eye surgery and eye pathology \parencite{magyezi2020eye, kaggwa2014ophthalmic}.
    }
    } 
\newglossaryentry{OCO}{
    name={\gls{oco}},
    description = {
    is a 
    % mid-level 
    healthcare professional in Uganda who provides essential eye care services, primarily at the secondary or district healthcare level \parencite{magyezi2020eye, kaggwa2014ophthalmic}.
    }
    } 
\newglossaryentry{subjective refraction}{
    name={Subjective refraction},
    description ={
    is the assessment of refractive status by a combination of spherical and cylindrical lenses to determine the best-corrected visual acuity with accommodation relaxed \parencite{kaur2023subjective}.
    }
    }


\newglossaryentry{objective refraction}{
    name={Objective refraction},
    description ={
    Objective refraction refers to the measurement of a person's refractive error using instruments without requiring their subjective feedback. Traditional methods like autorefractors are suitable for initial assessments but not as substitutes for subjective refraction \parencite{chauhan2022commentary}
    }
    }
\newglossaryentry{myopia}{
    name={Myopia},
    description ={
    commonly known as nearsightedness, is a prevalent refractive error characterized by the focal point of light falling in front of the retina \parencite{pavol2022myopia}
    }
    }

\newglossaryentry{hyperopia}{
    name={Hyperopia},
    description ={
    also known as hypermetropia or farsightedness, is a common refractive error prevalent in both children and adults, affecting individuals differently based on the degree of hyperopia, age, accommodative and convergence system status, and visual demands \parencite{khancase}
    }
    }

\newglossaryentry{astigmatism}{
    name={Astigmatism},
    description ={
    is a common refractive error characterized by distorted and irregular curvature of the cornea, leading to asymmetry in transverse coordinates and elliptical beam profiles \parencite{ali2022prevalence}
    }
    }

\newglossaryentry{presbyopia}{
    name={Presbyopia},
    description ={
    is an age-related vision disorder characterized by the progressive inability to focus on near objects, affecting individuals typically in their mid-40s and older \parencite{katz2021presbyopia}
    }
    }

\newglossaryentry{PEC}{
    name={\gls{pec}},
    description ={
    is the provison of appropriate , accessible and affordable care that meet patients eye care needs in a comprehensive and competent manner. \parencite{katz2021presbyopia}
    }
    }

\newglossaryentry{anisometropia}{
    name={Anisometropia},
    description ={
    condition where there is a significant difference in refractive error between the two eyes, leading to variations in visual acuity and potentially causing amblyopia. This condition is often analyzed in the context of myopia and hyperopia, where the degree of difference in refractive power can impact the visual development and overall eye health of an individual, especially in children \parencite{katz2021presbyopia}
    }
    }